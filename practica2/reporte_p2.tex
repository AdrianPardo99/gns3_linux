\documentclass[10pt]{article}
\usepackage[utf8]{inputenc}
\usepackage[spanish]{babel}
\usepackage{amsmath}
\usepackage{amsfonts}
\usepackage{amssymb}
\usepackage{graphics}
\usepackage{graphicx}
\usepackage[left=2cm,right=2cm,top=2cm,bottom=2cm]{geometry}
\usepackage{imakeidx}
\makeindex[columns=3, title=Alphabetical Index, intoc]
\usepackage{listings}
\usepackage{multicol}
\usepackage{changepage}
\usepackage{float}
\usepackage{cite}
\usepackage{url}
\usepackage{hyperref}
\usepackage{pdflscape}
\usepackage[document]{ragged2e}
\usepackage{xcolor,colortbl}

\hypersetup{
    colorlinks=true,
    linkcolor=blue,
    filecolor=magenta,
    urlcolor=blue,
}

\definecolor{Red}{rgb}{0.7,0,0}
\definecolor{LightCyan}{rgb}{0.88,1,1}
\definecolor{AquaCyan}{rgb}{0.2,1,0.5}
\definecolor{Gray}{gray}{0.85}
\definecolor{DarkBlue}{rgb}{0.1,0.1,0.5}

\definecolor{codegreen}{rgb}{0,0.6,0}
\definecolor{codegray}{rgb}{0.5,0.5,0.5}
\definecolor{codepurple}{rgb}{0.58,0,0.82}
\definecolor{backcolour}{rgb}{0.95,0.95,0.92}

\lstdefinestyle{mystyle}{
    backgroundcolor=\color{backcolour},
    commentstyle=\color{codegreen},
    keywordstyle=\color{magenta},
    numberstyle=\tiny\color{codegray},
    stringstyle=\color{codepurple},
    basicstyle=\ttfamily\footnotesize,
    breakatwhitespace=false,
    breaklines=true,
    captionpos=b,
    keepspaces=true,
    numbers=left,
    numbersep=5pt,
    showspaces=false,
    showstringspaces=false,
    showtabs=false,
    tabsize=3
}
\def\fillandplacepagenumber{%
 \par\pagestyle{empty}%
 \vbox to 0pt{\vss}\vfill
 \vbox to 0pt{\baselineskip0pt
   \hbox to\linewidth{\hss}%
   \baselineskip\footskip
   \hbox to\linewidth{%
     \hfil\thepage\hfil}\vss}}
\lstset{style=mystyle}

\lstset{
     literate=%
         {á}{{\'a}}1
         {í}{{\'i}}1
         {é}{{\'e}}1
         {ý}{{\'y}}1
         {ú}{{\'u}}1
         {ó}{{\'o}}1
         {ñ}{{\~n}}1
}


\title{Escuela Superio de Cómputo\\Instituto Politécnico Nacional\\Administración de Servicios en Red\\Practica 2\\Curso impartido por: Ricardo Martinez Rosales}

\author{Adrian González Pardo}

\date{\today}

\newcommand\tab[1][1cm]{\hspace*{#1}}

\begin{document}
\maketitle
\section{Descripción y Desarrollo}
Para desarrollar esta practica, previamente se estudio y se leyo acerca de los siguientes temas:
\begin{itemize}
  \item Webservice con Flask
  \item Sqlite en Python
  \item Enrutamiento estatico
  \item Modelo de desarrollo REST
  \item Uso de Interfaces Virtuales de Red para GNS3
  \item Creación, Modificación y Eliminación de usuarios en CISCO
  \item Configuración de protocolos para conexión remota (SSH y Telnet)
\end{itemize}
Con esto se desarrollo una aplicación REST que fuese capaz de procesar la información de tal modo que pudiese administrar los usuarios que se encuentran en la misma, por ello analizando las instrucciones que procesa HTTP se encontro que existen más peticiones como POST, PUT, DELETE en las cuales nos pudimos auxiliar para el desarrollo de la app. \\
Finalmente con la gracia y el deseo de realizar una reutilización del código se obto por crear algunos modulos en archivos de Python, para poder reutilizarlos más adelante.
\section{Modulos}
\lstinputlisting[language=python]{app/conecta.py}
\lstinputlisting[language=python]{app/databaseController.py}
\section{App}
\lstinputlisting[language=python]{app/web_server_create_user.py}

Finalmente con este código y debido a que no se desea realizar uso o crear un archivo html para esto, se procede con la realización de scripts en bash para el consumo de la aplicación
\section{Scripts}
\lstinputlisting[language=bash]{app/scrip_alta.sh}
\lstinputlisting[language=bash]{app/scrip_mod.sh}
\lstinputlisting[language=bash]{app/scrip_del.sh}
\section{Parte de demostración}
Para la demostración de como se desarrollo y una pequeña explicación se grabo un vídeo que se puede ver \underline{\href{https://youtu.be/adFZNEVpoVw}{aquí}} destacando y pidiendo disculpa por la cantidad de sueño que se expresa directamente o indirectamente durante la grabación y presentación.
\end{document}
